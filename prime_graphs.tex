\begin{description}
\item[Простим] відносно декартового добутку називається граф $G$, якщо він не тривіальний і з то що $G=G_1 \times G_2$ слідує, що або $G_1 \cong U$, або $G_2 \cong U$, де $U$ - тривіальний граф.
\end{description}

З визначення операції декартового добутку випливає, що кількість вершин графа $G=G_1 \times G_2$ складає $\vert V(G) \vert = \vert V(G_1) \vert \times \vert V(G_2) \vert$, а тому очевидно, що всі графи, у яких число вершин є простим - прості.
