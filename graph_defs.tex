\begin{description}
\item[Граф] Граф (неорієнтовний) $G = (V,E)$ складається із скінченної множини вершин $V$ і множини $E \subseteq {V \choose 2}$ пар $\{u,v\}$, $u \neq v$, які носять назву ребра.
  Зазвичай ребро $\lbrace u,v \rbrace$ позначають просто як $uv \in E$.
\item[Суміжні ребра] Якщо $uv \in E$, то кажуть, що $u$ та $v$ суміжні.
        \item[Нуль-граф] Граф з порожньою множиною ребер ($E=\emptyset$) називають нуль-графом і позначають $\AE$.
        \item[Порожній граф] Нуль-граф з $V=\emptyset$ називають порожнім.
        \item[Інцидентність ребра і вершини] Якщо $u \in V, k \in E$ і $u \in k$, то $u$ і $k$ називають інцидентними, а $u$ - кінцем $k$.
        \item[Суміжні ребра] Ребра $k,l \in E$ називають суміжними, якщо вони мають спільний кінець, тобто $k \cap l \ne \emptyset$.
        \item[Петлі] Ребра інцидентні лише одній вершині називаються петлями.
        \item[Кратні ребра] Ребра називаються кратними, якщо вони інцидентні одним і тим же вершинам.
        \item[Мультиграф] Граф, що містить кратні ребра, називають мультиграфом.
        \item[Псевдограф] Якщо мультиграф має петлі, тобто $\exists u \in V : uu \in E$, то такий граф ще називають псевдографом.
        \item[Звичайний граф] Граф, що не містить петель та кратних ребер, називають звичайним.
        \item[Сусіди вершини] Сусідами вершини $u \in V$ назавають множину вершин суміжних даній вершині і позначають $N(u)$.
        \item[Ступенем вершини] Кількість сусідів вершини $u$ позначають $deg(u) = \vert N(u) \vert$ і називають її локальним ступенем або просто ступенем.
        \item[Ізольована вершина] Вершина $u$, для якої виконується $deg(u) = 0$ ізольована.
        \item[Висяча вершина] Якщо вершина називається висячою, якщо вона я кінцем лише одного ребра.
        \item[Елементи графа] Вершини та ребра графа також називають його елементами.
        \item[Порядок графа] Число вершин $\vert V \vert$ - порядок графа.
        \item[Розмір графа] Число ребер $\vert E \vert$ - розмір графа.
        \item[Маршрут] Маршрутом $M$ у графі, що з'єднує вершину $v_1$ з вершиною $v_2$, є така послідовність вершин і ребер, що чергуються $$M=(v_1,e_1,v_2,e_2,\dots,v_{n-1},e_{n-1},v_n),$$ така, що кожні два сусідні ребра $e_{n-i}$ та $e_i}$ мають спільну інцидентну вершину $v_i$.
        \item[Замкнений маршрут] Маршрут у якого $v_1=v_n$, тобто початок одночасно є і кінцем, називають замкненим.
        \item[Ланцюг] Ланцюгом називається маршрут $M$, в якому кожне ребро зустрічається не більше одного разу.
        \item[Простий ланцюг] Простим називається ланцюг, будь-яка вершина якого зустрічається в ньому не більше одного разу.
        \item[Цикл] Замкнений ланцюг називають циклом. Якщо цей ланцюг є одночасно і простим, то цикл також називають простим.
        \item[Зв'язні вершини] Вершини $u$ та $v$ називаються зв'язними, якщо існує маршрут з кінцями в цих точках.
        \item[Зв'язний граф] Граф називають зв'язним, якщо будь-яка пара його вершин є зв'язною і незв'язним в протилежному випадку.
        \item[Зв'язність графу] Мінімальна кількість вершин, вилучення яких призводить до утворення незв'язного графа називається зв'язністю графа.
        \item[Компонента зв'язності] Зв'язний підграф графа $G$, який не є підграфом жодного іншого зв'язного підграфа графа $G$ називають компонентою зв'язності.
        \item[Міст] Мостом називається ребро вилучення якого призводить до збільшення компонент зв'язності.
        \item[Відстань між вершинами] Довжина найменшого ланцюга між вершинами звичайного графу називається відстанню між цими вершинами і позначається $d(u,v)$, де $u,v \in V$
        \item[Діаметр графа] Діаметром графа є величина $D(G) = \displaystyle{\max_{u,v \in V}d(u,v)}$
        \item[Частина графа] Граф $H(V^\prime,E^\prime)$, у якого $V^\prime \subseteq V$ та $E^\prime \subseteq E$ називається частиною графа $G(V,E)$.
        \item[Суграф] Якщо $V=V^\prime$, то частина графа називається суграфом.
        \item[Підграф] Частина графа $H=(V^\prime,E^\prime)$ графа $G=(V,E)$ називається підграфом, якщо справедливе $K^\prime=K\cap {V^\prime \choose 2}$, тобто $H$ містить усі ребра між вершинами в $V^\prime$, які також належать графу $G$.
        \item[$r$-регулярний граф] Граф називають $r$-регулярним, якщо для всіх його вершин $u \in V$ виконується $deg(u)=r$
        \item[Ізоморфні графи] Графи $G = (V,E)$ та $G^\prime = (V^\prime,E^\prime)$ називаються ізоморфними $G \cong G^\prime$, якщо існує бієкція $\varphi : E \to E^\prime$ така, що виконується $uv \in E \Leftrightarrow \varphi (u)\varphi (v) \in K^\prime$.
\end{description}
