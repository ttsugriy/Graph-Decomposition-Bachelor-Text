\begin{description}
\item[Граф] (неорієнтовний) $G = (V,E)$ складається із множини вершин $V$ і множини $E \subseteq {V \choose 2}$ пар $\{u,v\}$, $u \neq v$, які носять назву ребра. Якщо множина вершин скінченна, то граф називають скінченним, в іншому разі - нескінченним.
  Зазвичай ребро $\lbrace u,v \rbrace$ позначають просто як $uv \in E$.
\item[Інцидентність ребра і вершини] Якщо $u \in V, k \in E$ і $u \in k$, то $u$ і $k$ називають інцидентними, а $u$ - кінцем $k$.
\item[Суміжними] називаються ребра $k,l \in E$, якщо вони мають спільний кінець, тобто $k \cap l \ne \emptyset$.
\item[Петлями] називаються ребра інцидентні лише одній вершині.
\item[Кратними] називаються ребра, якщо вони інцидентні одним і тим же вершинам.
\item[Мультиграфом] називають граф, що містить кратні ребра.
\item[Псевдографом] називають мультиграф, що має петлі, тобто $\exists u \in V : uu \in E$.
\item[Звичайним] називають граф, що не містить петель та кратних ребер.
\item[Сусідами вершини] $u \in V$ називають множину вершин суміжних даній вершині і позначають $N(u)$.
\item[Степінь вершини] Кількість сусідів вершини $u$ позначають $deg(u) = \vert N(u) \vert$ і називають її локальним ступенем або просто ступенем.
\item[Ізольованою] називають вершина $u$, для якої виконується $deg(u) = 0$.
\item[Висячою] називається вершина, якщо вона я кінцем лише одного ребра.
\item[Елементами графа] називають його вершини та ребра.
\item[Порядком графа $G=(V,E)$] є кількість його вершин $\vert V \vert$.
\item[Розміром графа $G=(V,E)$] називають число його ребер $\vert E \vert$.
\item[Маршрутом] $M$ у графі, що з'єднує вершину $v_1$ з вершиною $v_2$, є така послідовність вершин і ребер, що чергуються $$M=(v_1,e_1,v_2,e_2,\dots,v_{n-1},e_{n-1},v_n),$$ така, що кожні два сусідні ребра $e_{n-i}$ та $e_i$ мають спільну інцидентну вершину $v_i$.
\item[Замкненим] називають маршрут у якого $v_1=v_n$, тобто початок одночасно є кінцем.
\item[Ланцюгом] називається маршрут $M$, в якому кожне ребро зустрічається не більше одного разу.
\item[Простим] називається ланцюг, будь-яка вершина якого зустрічається в ньому не більше одного разу.
\item[Циклом] називають замкнений ланцюг. Якщо цей ланцюг є одночасно і простим, то цикл також називають простим.
\item[Зв'язними] називаються вершини $u$ та $v$, якщо існує маршрут з кінцями в цих точках.
\item[Зв'язним] називають граф, якщо будь-яка пара його вершин є зв'язною і незв'язним в протилежному випадку.
\item[Зв'язність графу] Мінімальна кількість вершин, вилучення яких призводить до утворення незв'язного графа називається зв'язністю графа.
\item[Компонентою зв'язності] називають зв'язний підграф графа $G$, який не є підграфом жодного іншого зв'язного підграфа графа $G$.
\item[Мостом] називається ребро вилучення якого призводить до збільшення кількості компонент зв'язності.
\item[Відстанню між вершинами] називають довжину найменшого ланцюга між вершинами звичайного графа. Позначається $d(u,v)$, де $u,v \in V$.
\item[Діаметром графа] є величина $D(G) = \displaystyle{\max_{u,v \in V}d(u,v)}$.
\item[Частиною графа $G=(V,E)$] називають граф $H(V^\prime,E^\prime)$, у якого $V^\prime \subseteq V$ та $E^\prime \subseteq E$.
\item[Суграфом] графа $G=(V,E)$ називається частина графа $H=(V^\prime,E^\prime)$, якщо $V^\prime=V$.
\item[Підграфом] називається частина графа $H=(V^\prime,E^\prime)$ графа $G=(V,E)$, якщо справедливе $K^\prime=K\cap {V^\prime \choose 2}$, тобто $H$ містить усі ребра між вершинами в $V^\prime$, які також належать графу $G$.
\item[$r$-регулярним] називають граф $G=(V,E)$, якщо для всіх його вершин $u \in V$ виконується $deg(u)=r$.
\item[Ізоморфними $G \cong G^\prime$] низивають графи $G = (V,E)$ та $G^\prime = (V^\prime,E^\prime)$, якщо існує бієкція $\varphi : E \to E^\prime$ така, що виконується $uv \in E \Leftrightarrow \varphi (u)\varphi (v) \in K^\prime$.
\end{description}
