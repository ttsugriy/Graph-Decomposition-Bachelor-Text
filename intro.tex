Ми живемо в інформаційному суспільстві, про що свідчить важливість і вплив інформації на життя людей.
З появою Інтернет та гіпертекстових посилань інформація стала не лише більш доступною, але і зручнішою для використання.
Для моделювання процесів та структури інформації надзвичайно зручними є графи.
Для отриманих моделей можна використовувати алгоритми з теорії графів, що дозволяє аналізувати різні аспекти інформації.
Кількість людей, що мають доступ до Інтернет кожного року збільшується, як збільшується і кількість інформації, яку генерує людство.
Так згідно IBM\cite{web:ibm_soft_boost} кількість цифрової інформації в 2010 році буде подвоюватися кожні 11 годин.

Незважаючи на все ще діючий закон Мура, обчислювальні потужності не справляються з виникаючими задачами по обробці даних, і зважаючи на відставання в темпах від появи нової інформації, єдиним способом подолання проблем залишається підвищення ефективності існуючих алгоритмів.

Очевидно, що чим простіші задачі, тим легше їх можна вирішити.
Саме тому в даній роботі розглядається задача декомпозиції графів на прості підграфи і представлено найшвидший існуючий алгоритм для її розв'язання. 
