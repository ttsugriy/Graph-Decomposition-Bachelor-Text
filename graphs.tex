Граф (ненаправлений) $G = (V,E)$ складається із скінченної множини вершин $V$ і множини $E \subseteq {V \choose 2}$ пар $\{u,v\}$, $u \neq v$, які носять назву ребра.
Зазвичай ребро $\lbrace u,v \rbrace$ позначають просто як $uv \in E$.

Для графа $G = (V,E), u,v \in V$ використовують наступну термінологію:
\begin{itemize}
        \item Якщо $uv \in E$, то кажуть, що $u$ та $v$ суміжні
        \item Ребро $uu \in E$ називається петлею
        \item Ребра, що з'єднують одну й ту саму пару вершин, називають кратними (паралельними) ребрами
        \item Якщо $u \in V, k \in E$ і $u \in k$, то $u$ і $k$ називають інцидентними, а $u$ - кінцем $k$
        \item Ребра $k,l \in E$ називають інцидентними, якщо вони мають спільний кінець, тобто $k \cap l \ne \emptyset$
        \item Множину сусідів $u \in V$ позначають $N(u)$
        \item Кількість сусідів вершини $u$ позначають $deg(u) = \vert N(u) \vert$ і називають її порядком
        \item Вершина $u$, для якої виконується $deg(u) = 0$ ізольована
        \item Вершини та ребра графа також називають його елементами
        \item Число вершин $\vert V \vert$ - порядок графа
        \item Число ребер $\vert E \vert$ - розмір графа
\end{itemize}

Додатково вирізняють такі важливі графи:
\begin{enumerate}
        \item $\vert V \vert = n$ $E = {V \choose 2}$ (всі вершини з'єднані ребрами) - повний граф $K_n = (V,E)$
        \item Граф $G = (L+R,E)$ називають дводольним, якщо $V$ складається з двох множин $L$ і $R$, що не перетинаються, тобто $L \cap R = \emptyset$ і кожне ребро складається з вершин, одна з яких належить $L$, а друга - $R$. Якщо ж між усіма вершинами $L$ і $R$ існують всі ребра, то такий граф називають повним дводольним $K_{L,R}$, або $K_{m,n}$, якщо $\vert L \vert = m, \vert R \vert = n$.
        \item Узагальненням повного дводольного є повний $k$-дольний граф $K_{n_1,\dots,n_k}$ у якого:
        \begin{itemize}     
        \item $V=V_1+\dots+V_k$ та $V_i \cap V_j = \emptyset$ для всіх $i \ne j$
        \item $\vert E_i \vert = n_i (i=1,\dots,k)$
        \item $E = \lbrace uv : xu \in E_i, v \in E_j, i \ne j \rbrace$
        \end{itemize}
        \item Гіперкубом $Q_n$ називається граф, вершинами якого є всі послідовності 0,1 довжини $n$, тобто $\vert E \vert = 2^n$. Між усіма вершинами $u$ та $v$ існують ребра, якщо послідовності 0,1 цих вершин відрізняються тільки у одному місці.
\end{enumerate}

Додаткові означення:
\begin{itemize}
        \item Шляхом $P_n$ в графі є послідовність вершин, що не повторяються $u_1,u_2,\dots,u_n$, таких, що $u_iu_{i+1} \in E, i = \overline {1,\dots,n-1}$.
        \item Циклом $C_n$ графу є шлях $P_n$ у якого $u_nu_1 \in E$.
        \item Граф $H(V^\prime,E^\prime)$, у якого $V^\prime \subseteq V$ та $E^\prime \subseteq E$ називається підграфом графа $G(V,E)$.
        \item Якщо $\exists u \exists v : \vert \lbrace u,v \rbrace \vert > 1, u,v \in V$, тобто між двома вершинами існує більше одного ребра, то такий граф називають мультиграфом.
        \item Якщо мультиграф має петлі, тобто $\exists u \in V : uu \in E$, то такий граф ще називають псевдографом.
        \item Графи $G = (V,E)$ та $G^\prime = (V^\prime,E^\prime)$ називаються ізоморфними $G \cong G^\prime$, якщо існує бієкція $\varphi : E \to E^\prime$ така, що виконується $uv \in E \Leftrightarrow \varphi (u)\varphi (v) \in K^\prime$.
\end {itemize}

