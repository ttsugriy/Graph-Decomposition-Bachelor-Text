\begin{description}
        \item[Повний] $\vert V \vert = n$ $E = {V \choose 2}$ (всі вершини з'єднані ребрами) - повний граф $K_n = (V,E)$
        \item[Дводольний] Граф $G = (L \cup R,E)$ називають дводольним, якщо $V$ складається з двох множин $L$ і $R$, що не перетинаються, тобто $L \cap R = \emptyset$ і кожне ребро складається з вершин, одна з яких належить $L$, а друга - $R$. Якщо ж між усіма вершинами $L$ і $R$ існують всі ребра, то такий граф називають повним дводольним $K_{L,R}$, або $K_{m,n}$, якщо $\vert L \vert = m, \vert R \vert = n$.
        \item[Повний $r$-дольний] Узагальненням повного дводольного є повний $k$-дольний граф $K_{n_1,\dots,n_k}$ у якого:
        \begin{itemize}     
        \item $V=V_1 \cup \dots \cup V_k$ та $V_i \cap V_j = \emptyset$ для всіх $i \ne j$
        \item $\vert E_i \vert = n_i (i=1,\dots,k)$
        \item $E = \lbrace uv : xu \in E_i, v \in E_j, i \ne j \rbrace$
        \end{itemize}
        \item[Гіперкуб] Гіперкубом $Q_n$ називається граф, вершинами якого є всі послідовності 0,1 довжини $n$, тобто $\vert E \vert = 2^n$. Між усіма вершинами $u$ та $v$ існують ребра, якщо послідовності 0,1 цих вершин відрізняються тільки у одному місці.
\end{description}
