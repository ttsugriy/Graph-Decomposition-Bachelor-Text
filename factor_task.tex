Проблема факторизації графів вже протягом 50 років активно піднімається вченими із всього світу.
Існування єдиного способу декомпозиції скінченних графів і широкого класу нескінченних графів відносно декартового добутку вперше показав Герт Сабідуссі у 1959 році\cite{sabidussi59}.
Незалежно від нього даний факт для скінченних графів довів Візінг у 1966 році.
Але доведення Сабідуссі не було алгоритмічним, а практичне застосування було неможливим.
Лише у 1985 році Файгенбаум запропонував алгоритм декомпозиції графа $G$ на прості підграфи з часовою складністю $O(n^4\sqrt n)$, де $n=\vert V \vert$.

Незалежно від Файгенбаума, Вінклер у 1987 році, використовуючи об'єднану роботу Грехема про канонічне вкладення графа $G$ в декартовий добуток, опублікував алгоритм із часовою оцінкою $O(n^4)$
Подальше доопрацювання алгоритму Вінклера Ауренхаммером, Хагуаром та Хохштрассером дозволило покращити часову оцінку до $O(mn+n^2\log^2n$, а оцінку пам'яті до $O(n^2)$, де $m=\vert E \vert$.

У 1992 році Федер запропонував алгоритм із часовою оцінкою $O(nm)$ та оцінкою пам'яті $O(m)$.

В даній же роботі робиться аналіз алгоритму факторизації графів з часовою оцінкою $O(m\log n$ та оцінкою пам'яті $O(m)$, вперше запропонований Ауренхаммером, Хагауером та Імріхом у 1992 році\cite{aurenhammer92}.
