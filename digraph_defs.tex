
Більшість означень неорієнтованих графів є аналогічними для орграфів, за виключенням використання дуг замість ребер.

\begin{description}
  \item[Орієнтований граф] (орграф) $\overrightarrow G=(V,E)$ складається з множини вершин $V$ і множини впорядкованих пар $E \subseteq V^2$, які називають орієнтованими ребрами або дугами, при чому вершини $u$ та $v$ дуги $e=(u,v)$ називаються початком ($u=k^-$) і кінцем ($v=k^+$) відповідно.
    \begin{figure}[h]
      \centering
      \input fig_digraph
      \caption{Приклад орграфа}
      \label{fig:digraph}
    \end{figure}
  \item[Вхідним степенем вершини] $u$ називають число дуг з кінцем у даній вершині - $deg^+(u)=\vert\lbrace e \in E : e^+=u\rbrace \vert$.
  \item[Вихідним степенем вершини] $u$ є число дуг з початком у даній вершині - $deg^-(u)=\vert \lbrace e \in E : e^-=u \rbrace \vert$.
  \item[Слабкозв'язним орграфом] називається граф $\overrightarrow G=(V,E)$, якщо відповідний цьому графу неорієнтований граф $G=(V,E)$ є зв'язним.
  \item[Сильнозв'язним орграфом] називається граф $\overrightarrow G$, якщо існує орієнтований маршрут між будь-якими вершинами $u,v \in V$.
\end{description}
