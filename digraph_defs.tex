
Визначення орієнтованого маршруту, ланцюга та циклу не наводяться, оскільки вони є аналогічними визначенням для неорієнтованого графа за виключенням використання дуг замість ребер.

\begin{description}
  \item[Орієнтований граф] Орієнтований граф (орграф) $\overleftarrowG=(V,E)$ складається з множини вершин $V$ і множини впорядкованих пар $E \subseteq V^2$, які називають орієнтованими ребрами або дугами, при чому вершини $u$ та $v$ дуги $e=(u,v)$ називаються початком ($u=k^-$) і кінцем ($v=k^+$) відповідно.
  \item[Вхідний степінь вершини] Вхідним степенем вершини $u$ називають число дуг з кінцем у даній вершині - $deg^+(u)=\vert\lbrace e \in E : e^+=u\rbrace \vert$.
  \item[Вихідний степінь вершини] Вихідним степенем вершини $u$ є число дуг з початком у даній вершині - $deg^-(u)=\vert \lbrace e \in E : e^-=u \rbrace \vert$.
  \item[Зв'язний орграф] Орграф $\overrightarrow G=(V,E)$ називається зв'язним, якщо відповідний цьому графу неорієнтований граф $G=(V,E)$ є зв'язним.
  \item[Сильнозв'язний орграф] Орграф $\overrightarrow G$ називається сильнозв'язним, якщо існує орієнтований маршрут між будь-якими вершинами $u,v \in V$.
\end{description}
