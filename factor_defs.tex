\begin{description}
\item[Декомпозованим] на прості підграфи (у сенсі декартового добутку) зв'язний граф $G$ може бути тоді і тільки тоді, якщо існує множина простих графів $\lbrace G_i \vert i=\overline{1,\dots,n} \rbrace$, таких що $G \cong \displaystyle\prod_{i=0}^n G_i$.
\item[Пошуком в ширину] (англ. Breadth-First-Search, скорочено BFS)  на графі $G$ називається спосіб пошуку вершини графу, при якому спочатку перевіряється початкова вершина та її сусіди, потім дана процедура рекурсивно повторюється для усіх її сусідів.
\item[BFS-відсортованим] називають граф, вершини якого відсортовані згідно порядку відвідування при застосуванні алгоритму пошуку в ширину.
Тобто для вершин графу справедливе твердження, що якщо $d_G(v_0,v_i)>d_G(v_0,v_j)$, то $i>j$.
\item[BFSNUM(v)] є функція, що для вершини $v$ графа $G$ ставить у відповідність її порядковий номер при проході $G$ використовуючи $BFS$ алгоритм.
Для зручності надалі в роботі буде вважатися, що вершинами графа є числа із $\lbrace 0,1,2, \dots, n \rbrace$, причому $v=BFSNUM(v)$.
\item[Маркувальною функцією] називають ін'єктивну функцію $f : V\to\lbrace 0,1,2,\dots,n \rbrace^t$ для деяких $n \geq 0, t \geq 1$.
\item[Маркером вершини $v$] називають $t$-кортеж $f(v)$.
\item[Маркованим] називають граф з маркувальною функцією.
\item[Маркованим графом добутків (МГД)] $\displaystyle G=\prod_{i=1}^t G_i$ називають граф, маркерами вершин якого є елементи $V_1 \times \dots \times V_t$, де $V_i = V(G_i), i=\overline{1 \dots t}$.
\item[Оптимально маркованим графом добутків (ОМГД)] є граф $G$, якщо всі підрафи його маркованого добутку графів є простими.
\item[Початком МГД] є вершина $v_0$ з маркером $\lbrack 0,\dots,0 \rbrack$.
\item[Маркуючою добутки] називається маркувальна функція $f$ для МГД.
\item[$k$-розрізом] МГД є підграф, вершини, якого мають наступні маркери:$$\lbrack i_1,\dots,i_{k-1},j,i_{k+1},\dots,i_t,j \in V_k,$$ де $\lbrack i_1,\dots,i_{k-1},i_{k+1},\dots,i_t \rbrack$ є фіксованими $(t-1)$-кортежами в $V_1 \times \dots \times V_{k-1} \times V_{k+1} \dots \times V_k$.
\item[$k$-одиничним розрізом (в $f$, або відносно $f$)] називається $k$-розріз, який включає початок $v_0$.
\item[Вершиною одиничного розрізу] називається вершина з маркером зі щонайбільше одною ненульовою компонентою. В іншому випадку вершина називається вершиною неодиничного розрізу.
\item[Ребро $uv$ знаходиться в $i$-му факторі] якщо маркери $f(u)$ та $f(v)$ відрізняються виключно однією компонентою, тобто $f(u,i) \ne f(v,i)$ і $f(u,j) = f(v,j)$ для $j \ne i$.
\item[Еквівалентними відносно $f$] називають ребра $e$ та $e^\prime$, якщо обидва ребра знаходяться у однаковому факторі, і позначають $e \equiv_f e^\prime$.
\item[Максимальним рівнем] графа $G$ для вершини $v_0$ називають $d_{max}=\max_{v \in V}d_G(v_0,v)$.
\item[Рівнем $k$] графа $G$ є множина вершин $v$, для яких $d_G(v_0,v)$.
\item[$G(k)$] позначають підграф графа $G$, що складається з усіх вершин, що знаходяться у рівні $i$ для $i \le k$.
\item[Маркованим частковим графом добутку (МЧГД)] є граф $G$, якщо існує МГД $H$ з маркуючою функцією $h$ та ціле число $k$, таке що $H(k) \cong G$ і $f$ є обмеженням $h$ для $H(k)$.
\item[Добуток-сумісною] називають маркувальну функцію $f$ МГД $G$, якщо має місце наступне:
  \begin{enumerate}
    \item Існує точно одна вершина $v_0$ з $f(v_0,k)=0$ для всіх $k$, $1 \le k \le t$.
    \item Маркери інцидентних вершин відрізняються лише однією компонентою.
    \item Якщо тільки $k$-а компонента вершини $v$ відрізняється від нуля, тоді всі вершини на будь-якому коротшому шляху від $v$ до $v_0$ (окрім $v_0$) мають цю властивість.
Очевидно, що всі ці вершини разом з $v_0$ формують зв'язний граф, який називають $k$-м одиничним розрізом $f$ і, аналогічно попередньому визначенню, кажуть, що $uv$ знаходяться в $i$-му факторі якщо $f(u)$ та $f(v)$ різняться виключно $i$-ою компонентою.
    \item $f(v,k)=BFSNUM(v)=v$ для будь-якої вершини$v$ одиничного розрізу.
    \item Для будь-якої вершини $v$ та кожного $k$ вершина $u=f(v,k)$ існує і є вершиною $k$-одиничного розрізу.
    \item Для кожної вершини $v$ існує ребро $vu$ в факторі $k$ тоді і тільки тоді, якщо існує ребро $f(v,k)f(u,k)$ в факторі $k$.
  \end{enumerate}
  \item[Частково добуток-сумісною] називається маркувальна функція $f$ для якої виконуються вимоги добуток-сумісної функції з $1)$ по $5)$ а $6)$ замінене наступним:
    \begin{enumerate}
      \setcounter{enumi}{6}
      \item Для кожної вершини в рівні $i < d_{max}$ справедливе $6)$, і для кожної вершини $v$ в рівні $d_{max}$ існує ребро $uv$ в факторі $k$ тоді і тільки тоді, якщо існує ребро $f(v,k)f(u,k)$ в факторі $k$ і $f(u,k)$ в не вищому рівні ніж $f(v,k)$.
    \end{enumerate}
    \item[$Index(u,v,i)$] $:=\lbrace j \vert f(u,j) \ne 0 \lor f(v,j) \ne 0 \lor j = i \rbrace$.
    \item[Функція Об'єднання$(u,v;i)$] є функцією Об'єднання $: f \to f^\prime$,\\
      де Об'єднання$(f(w))=f^\prime(w)$, де в свою чергу:
      \begin{enumerate}
        \item $f^\prime(w,j):=f(w,j)$, якщо $j \not\in Index$.
        \item $f^\prime(w,i):=x$, якщо $f(x,j)=f(w,j)$ для всіх $j \in Index$ та $f(x,j) = 0$ для $j \not\in Index$.
        \item $f^\prime(w,j):=$невизначено, якщо $j \in Index$ $j \ne i$.
      \end{enumerate}
      \item[Відношення ``$\triangleleft$''] Для маркувальних функцій добутку або часткового добутку $f$ та $g$ графа $G$, що містять спільний початок, відношення ``$\triangleleft$'' визначене наступним чином: $f \triangleleft g$, якщо кожна вершина одиничного розрізу в $f$ є вершиною одиничного розрізу в $g$.
\end{description}
