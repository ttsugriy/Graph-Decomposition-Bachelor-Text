\begin{lemma}
  Нехай $G=(V,E)$ МГД з маркувальною функцією $f$, такий що $\displaystyle G=\prod_{i=1}^t G_i$ для $t$ графів $G_i$.
Тоді мають місце наступні твердження:
\begin{enumerate}
\item Маркери двох інцидентних вершин різняться точно однією компонентою.
\item Якщо маркери вершин $u$ та $v$ відрізняються у $k$ компонентах, то має місце $d_G(u,v) \ge k$.
\item Для кожної вершини $v$ з $f(v,i) > 0$ для деякого $i,1 \le i \le t$ існує вершина $u$ з $d_G(u,v_0) < d_G(v,v_0)$, $f(u,i)=0$, $f(u,j)=f(v,j)$ для $i \ne j$.
\item Якщо для деякої вершини $v$ маркер $f(v)$ містить $k$ ненульових компонент, тоді $G$ має щонайменше $2^k$ вершин $u_i$, $d_G(u_i,v_0) \le d_G(v,v_0)$, для $1 \le i \le 2^k$.
\item $d_G(v,v_0)=\sum_{i=1}^t d_{G_i}(0,f(v,i))$.
\item Нехай $vu$, $uw$, $wx$, $xv$ є квадратом (з або без діагоналей), тоді $vu$ та $wx$ знаходяться в такому ж факторі, що і $uw$ та $xv$.
\item Нехай $vu$ та $vw$ є ребрами у різних факторах, тоді існує $x$ та ребра $xu$ і $xw$ в $G$, причому ні $xv$, ні $uw$ не є ребрами графа $G$.
\item Нехай $v$ є вершиною в рівні $k \ge 2$ і $u$ та $w$ є вершинами в рівні $k-1$. Якщо $vu$ та $vw$ є ребрами в різних факторах, тоді існує вершина $x$ в рівні $k-2$ інцидентна і $u$, і $w$.
\end{enumerate}
\end{lemma}

\begin{lemma}
  Нехай $G$ є МГД, $f$ - маркувальна функція з $t$ компонентами, $v$ - неодинична вершина розрізу в рівні $k$, для деякого $k$, $1 \le k \le d_{max}$. Тоді:
  \begin{enumerate}
    \item Існує пара вершин $u$, $w$ в рівні $k-1$ інцидентних $v$ з наступними властивостями:
      \begin{enumerate}
        \item $u$ та $w$ різняться точно в двох компонентах, скажімо $i$ та $j$, а також
        \item $f(v,l)=f(u,l)=f(w,l)$ для $1 \le l \le t$, $l \ne i$, $l \ne j$, $f(v,i)=max \lbrace f(u,i),f(w,i) \rbrace$, $f(v,j)=max \lbrace f(u,j),f(w,j) \rbrace$.
      \end{enumerate}
    \item Кожна пара вершин $u$, $w$ в рівні $k-1$ інцидентна $v$, що задовольняє $(a)$ також задовольняє і $(b)$.
    \item Для кожної вершини $u$ в рівні $k-1$ інцидентної $v$, існує вершина $w$ в рівні $k-1$ інцидентна $v$, що задовольняє $(a)$ і $(b)$.
  \end{enumerate}
\end{lemma}

\begin{lemma}
  Маркований граф $G$ є МГД тоді і тільки тоді, якщо маркувальна функція $f$ є добуток-сумісною. 
\end{lemma}

\begin{lemma}
  Маркований граф $G$ є МЧГД тоді і тільки тоді, коли маркувальна функція $f$ є частково добуток-сумісною.
\end{lemma}

\begin{lemma}
  Нехай $G$ є МГД (відповідно МЧГД) з маркувальної функцією $f$, тоді для всіх $u,v \in G$ і кожного $i, 1 \le i \le t$, маркований граф $G$, отриманий з Об'єднання$(u,v;i)$ є МГД (відповідно МЧГД).
\end{lemma}

\begin{lemma}
  Якщо $f \triangleleft g$ тоді $e \equiv_f e^\prime$ означає $e \equiv_g e^\prime$. Тобто ``$\equiv_g$'' є ширшим відношенням еквівалентності на $E$ ніж ``$\equiv_f$''.
\end{lemma}

\begin{lemma}
  Нехай $Factor(i):=\lbrace j \vert$ якщо $e$ в факторі $j$ в $f$, тоді $e$ в факторі $i$ в $g \rbrace$. Якщо $f \triangleleft g$ та $g(u,i)=g(v,i)$, тоді $f(u,i^\prime)=f(v,i^\prime)$ $\forall i^\prime \in Factor(i)$.
\end{lemma}
