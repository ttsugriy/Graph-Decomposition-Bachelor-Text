Теорія графів є однією з центральних тем дискретної математики, яка дивовижним чином поєднує практику з теорією, наочність та заплутаність методів, історію і сучасність.
Її застосування особливо помітне в теоріях інформатики та комунікацій, плануванні доріг та бізнес процесів тощо.
Виникнення теорії графів приписують Леонарду Ойлеру (1707-1783) та його роботі присвяченій семи мостам Кьонігсберга (Рис. \ref{koenigsberg_graph}).
\begin{figure}[h]
        \centering
                \input koenigsberg_graph
        \caption{Граф, що моделює Кьонігсбергські мости}
        \label{koenigsberg_graph}
\end{figure}
Будучи по суті звичайними множинами із визначеними бінарними відношеннями, графи дозволяють моделювати процеси будь-якої складності.
Разом із простотою графи є надзвичайно зручні для візуального представлення, а тому часто дозволяють людині візуально розв'язавши задачу, формалізувати отриманий результат у зручному для подальшої обробки мові.

\newpage
\subsection{Графи}
\input graphs

\newpage
\subsection{Орграфи}
\input digraphs

c\newpage
\subsection{Операції над графами}
\input graph_ops

\newpage
\subsection{Прості графи}
\input prime_graphs
