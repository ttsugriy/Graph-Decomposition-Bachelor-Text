Теорія графів є однією з центральних тем дискретної математики, яка дивовижним чином поєднує практику з теорією, наочність та заплутаність методів, історію і сучасність. Її застосування особливо помітне в теоріях інформатики та комунікацій, плануванні доріг та бізнес процесів тощо. Будучи по суті звичайними множинами із визначеними бінарними відношеннями, графи дозволяють моделювати процеси будь-якої складності. Разом із простотою графи є надзвичайно зручні для візуального представлення, а тому часто дозволяють людині візуально розв'язавши задачу, формалізувати отриманий результат у зручному для подальшої обробки мові.
\newpage
\subsection{Графи}
\input graphs

\newpage
\subsection{Орграфи}
\input digraphs

\newpage
\subsection{Операції над графами}
\input graph_ops

\newpage
\subsection{Прості графи}
\input prime_graphs
